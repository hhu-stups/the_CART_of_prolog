\section{Einleitung}

Das Schlagwort maschinelles Lernen hat durch die heutzutage zur Verfügung stehende Rechenleistung,
immer mehr Aufmerksamkeit gefunden. Zudem ist durch das Internet die Sammlung von Daten sehr viel leichter
geworden und es stehen ausreichend Daten zur Verfügung, die analysiert werden können.
Um diese Daten zu verarbeiten werden gerne Klassifikationsalgorithmen benutzt. Diese finden nicht nur
Verwendung im Bereich des Data-Mining~\cite{sharma2016survey}, sondern auch in klinischen Studien~\cite{lewis2000introduction}
und der psychologischen Forschung~\cite{Strobl2009-hw}.
In diesen Feldern wird für die Klassifikation ein sogenannter Entscheidungsbaum benutzt.
Entscheidungsbäume sind in den eben genannten Gebieten beliebt, da sie sich leicht grafisch darstellen lassen
und die Regeln, nach denen klassifiziert wird, ablesbar und leicht verständlich sind.
Die zwei bekanntesten Algorithmen zur Erstellung von Entscheidungsbäumen sind schon lange im Umlauf.
Der CART-Algorithmus~\cite{breiman1984classification} wurde von Breiman im Jahr 1984 und C4.5, von Quinlan, im Jahr 1993~\cite{quinlan1993c45}
vorgestellt. CART lässt sich sogar für sogenannte Regressionsprobleme verwenden, wodurch für eine Eingabe auch ein stetiger Wert
vorhergesagt werden kann anstatt einer Klasse. Dieser Algorithmus ist folglich auch in verschiedenen Bibliotheken für maschinelles Lernen,
in verschiedenen Programmiersprachen, implementiert. Allerdings nicht in Prolog.

Prolog~\cite{Colmerauer1993TheBO} ist eine logische Programmiersprache und spielt eine große Rolle beim Bau von Expertensystemen~\cite{merritt2012building}
und wird heutzutage gerne im Bereich der künstlichen Intelligenz~\cite{shoham2014artificial} und diversen anderen Feldern~\cite{806816,wicaksono2016relational} eingesetzt.
Diese Programmiersprache besitzt keine Standard Bibliothek für Algorithmen des maschinellen Lernens, trotz seiner wichtigen Rolle in der künstlichen Intelligenz.
Da der CART-Algorithmus rekursiv ist, macht es durchaus Sinn diesen Algorithmus, in Prolog umzusetzen, da diese Programmierspache
hervorragend mit Rekursion umgehen kann. Zusätzlich würde eine Implementierung eines Entscheidungsbaums in Prolog zeigen, ob es sinnvoll ist
Machine-Learning-Algorithmen in puren Prolog zu implementieren.

In dieser Arbeit soll untersucht werden, wie gut eine pure Prolog Implementierung des Entscheidungsbaum Algorithmus
CART, im Vergleich zu den weit verbreiteten Referenzimplementierungen aus Python und R ist.
Dafür wird zunächst das benötigte Wissen über die benutzten Algorithmen und Prolog eingeführt.
Daraufhin wird die Implementierung und anschließend die Ergebnisse eines Performancevergleichs vorgestellt.