\section{Experiment}
In diesem Kapitel werden zunächst die nötigen Vorbereitungen für das Experiment in \cref{sec:Vorbereitung} vorgestellt.
In dem \cref{sec:Performance} wird die Präzision und die Ausführungszeit, der in dieser Arbeit vorgestellten Implementierung in 
SWI-Prolog~\cite{wielemaker:2011:tplp} und SICSTus-Prolog~\cite{carlsson1988sicstus},
mit Implementierungen aus den R Bibliotheken \enquote{rpart}~\cite{therneau2015package} und \enquote{randomForest}~\cite{rcolorbrewer2018package}
und der Python Bibliothek \enquote{scikit-learn}~\cite{scikit-learn}.

\subsection{Vorbereitung} \label{sec:Vorbereitung}
Zunächst wird die Art der Datenbeschaffung vorgestellt. Daraufhin werden die benutzten Metriken erklärt.
Zum Schluss werden die Implementierungen aus den Referenzimplementationen exemplarisch vorgestellt.

\subsubsection{Datenbeschaffung}
Um die Ergebnisse der einzelnen Implementierungen miteinander vergleichen zu können wurden jeweils fünf Train-Test-Aufteilungen
zu den einzelnen Datensätzen erstellt und in CSV-Dateien abgespeichert. Die Daten werden über scikit-learn heruntergeladen
und anschließend mithilfe der Bibliothek \enquote{pandas}~\cite{mckinney2011pandas} als CSV Datei abgespeichert.
Für Train-Test-Aufteilungen bietet scikit-learn die Funktion \texttt{train\_test\_split} an. Diese zerlegt einen übergebenen
Datensatz, in einem selbst bestimmbaren Verhältnis, in Trainings und Test Daten. Ich habe mich für 70\% der Daten
im Trainings- und 30\% der Daten im Test-Datensatz entschieden. Um die Zerlegung zu replizieren,
kann sogar ein Seed für den Zufallsgenerator übergeben werden. \cref{lst:daten} zeigt beispielhaft die Erstellung
einer Train-Test-Aufteilung. Für alle fünf Aufteilungen eines Datensatzes wurden jeweils die Seeds 18, 19, 20, 21 und 22 verwendet.
\begin{lstlisting}[
  float, caption={Train-Test-Split and convert to csv},
  label={lst:daten}, language=Python
]
iris_feature_train1, iris_feature_test1, iris_target_train1, iris_target_test1 =
        train_test_split(iris['data'],iris['target'],test_size=0.3,random_state=18)

df = pd.DataFrame(data=iris_feature_train1, columns = iris['feature_names'])
df['class'] = iris_target_train1
df.to_csv('iris_train1.csv', sep = ',', index = False)

df = pd.DataFrame(data=iris_feature_test1, columns = iris['feature_names'])
df['class'] = iris_target_test1
df.to_csv('iris_test1.csv', sep = ',', index = False)
\end{lstlisting}

\subsubsection{Zeitmessung}
Der Kern des Experiments soll es sein, die gemessene Trainingszeit der Implementierungen miteinander zu vergleichen.
Dafür wird in diesem Unterkapitel die Zeitmessung vorgestellt.

Die Zeitmessung ist in allen Implementierungen analog. Es wird eine vorgefertigte Methode für Zeitstempel
vor und nach dem Aufruf der Methode, die den Algorithmus durchführt, benutzt.
Daraufhin werden die Zeitstempel subtrahiert um die vergange Zeit zu erhalten oder die Methode für
den Zeitstempel hat bereits eine Ausgabe für den Abstand zum letzten Aufruf.
Letzteres ist der Fall in der Prolog Implementierung. Das eingebaute Prädikat \texttt{statistics/2}
liefert in seiner zweiten Variable, als zweiten Wert die Zeit zum letzten Aufruf in Millisekunden.
Das ist exemplarisch in \cref{lst:tree-pro} implementiert.

\begin{lstlisting}[
  float, caption={Prolog implementation of \texttt{create\_tree/2}},
  label={lst:tree-pro}, language=Prolog
]
create_tree(File, Tree, Time) :-
    csv_read_file(File, [H|T]),
    get_attribute_list(H, Attribute),
    fit_format(T, Attribute, Data),
    statistics(walltime, _),
    induce_tree(Data, Attribute, Tree),
    statistics(walltime, [_,Time]).
\end{lstlisting}

In der R Implementierung wird mit der Bibliothek \enquote{tictoc}~\cite{izrailev2014tictoc} gearbeitet.
Hier wird der Algorithmus Aufruf von zwei Aufrufen eingeschlossen, wobei der zweite Aufruf die vergange Zeit
auf der Konsole in Sekunden ausgibt und in ein Log schreibt. Das ist exemplarisch in \cref{lst:class-r} dargestellt.
Allerdings kann diese Implementierung nur bis auf 10 Millisekunden genau messen.
\begin{lstlisting}[
  float, caption={R implementation of \texttt{train.test.classification}},
  label={lst:class-r}, language=R
]
train.test.classification <- function(trainData, testData) {
  train <- read.csv(trainData)
  train$class <- as.factor(train$class)
  tic(trainData)
  fit <- rpart(class ~ ., train)
  toc(log = TRUE)
  test <- read.csv(testData)
  prediction <- predict(fit, newdata = test, type = "class")
  true <- prediction == test$class
  precision <- sum(true) / length(true)
  print(c("Precision:", precision))
  return(precision)
}
\end{lstlisting}

In der Python Implementierung wird mit der eingebauten \enquote{time} Bibliothek gearbeitet.
Hier wird vor und nach der Ausführung die Summe aus System und CPU-Zeit gemessen und anschließend
voneinander subtrahiert. Das ist exemplarisch in \cref{lst:tree-py} implementiert.
Hier ist die Präzision der Zeitmessung um die 16 Millisekunden begrenzt.

\begin{lstlisting}[
  float, caption={Python implementation of \texttt{train\_test\_classification}},
  label={lst:tree-py}, language=Python
]
def train_test_classification(feature_train,target_train,
                                   feature_test,target_test):
    t1 = time.process_time()
    clf = tree.DecisionTreeClassifier()
    clf = clf.fit(feature_train,target_train)
    t2 = time.process_time()
    trainTime = t2 - t1
    print('Train Time:       {time} sec.'.format(time=trainTime))
    preds = clf.predict(feature_test)
    prec = precision(preds, target_test)
    print('Precision:        {p}'.format(p = prec))
    return trainTime, prec
\end{lstlisting}

\subsubsection{Metriken}

Nun geht es darum die benutzten Metriken in der Implementierung vorzustellen.
Als Performance Metrik für die Klassifikation, habe ich mich für die einfachste entschieden,
die Genauigkeit, wie in \cref{eq:precision} dargestellt. Hier steht \(p_i\) für die i-te Vorhersage und nimmt den Wert 1 an,
wenn richtig vorhergesagt wurde oder 0, wenn nicht.
Im Fall der Regression habe ich mich für den \enquote{Mean-Squared-Error} \cref{eq:mse},
als Metrik für die Vorhersage, entschieden.
Zusätzlich wird für die Zeitmessung der \enquote{Standard Error of Mean} \cref{eq:sem} berechnt.
\begin{align}
  Precision               & =  1/n * \sum_{i=1}^n p_i  \label{eq:precision} \\
  Mean Squared Error       & = 1/n * \sum_{i=1}^n (y_i - f(x_i))^2 \label{eq:mse} \\
  Standard Error of Mean       & = o/\sqrt{n} \label{eq:sem} \\
  o                        & = \sqrt{(\sum_{i=1}^n (t_i - \mu)^2)/n-1 } 
  \,\text{,}
\end{align}
Hierbei ist \(y_i\) der i-te tatsächliche Wert und \(f(x_i)\) der i-te vorhergesagte Wert, \(t_i\) die i-te Zeitmessung und \(\mu\) das arithmetische Mittel.
Die Implementierung der Präzision ist für SWI- und SICSTus-Prolog in \cref{lst:prezi-pro}, für R in \cref{lst:class-r} (als Variable \enquote{precision})
und für Python in \cref{lst:prezi-py} dargestellt.
\begin{lstlisting}[
  float, caption={Prolog implementation of \texttt{precision/3}},
  label={lst:prezi-pro}, language=Prolog
]
%! precision(+Klassen:list, +Pred:list, -Gen:double) is det
%
%  Berechnet den Anteil der Elemente,
%  die in beiden Listen miteinander uebereinstimmen
precision(Klassen, Pred, Gen) :-
    precision(Klassen, Pred, 0, 0, Gen).
precision([], [], Len, Acc, Gen) :- !, Gen is Acc / Len.
precision([KH|KT], [PH|PT], Len, Acc, Gen) :-
    NewLen is Len + 1,
    (KH == PH ->
        NewAcc is Acc + 1,
        precision(KT, PT, NewLen, NewAcc, Gen)
    ;   precision(KT, PT, NewLen, Acc, Gen)).
\end{lstlisting}

\begin{lstlisting}[
  float, caption={Python implementation of \texttt{precision}},
  label={lst:prezi-py}, language=Python
]
def precision(preds, test):
    l = len(preds)
    akk = 0
    for i in range(l):
        if(preds[i] == test[i]):
            akk += 1
    return akk / l
\end{lstlisting}

Die Implementierung des Mean Squared Error ist für SWI- und SICSTus-Prolog in \cref{lst:mse-pro}, für R in \cref{lst:mse-r} (als Variable \enquote{mse})
und für Python in \cref{lst:mse-py} dargestellt.

\begin{lstlisting}[
  float, caption={Prolog implementation of \texttt{mean\_sq\_error/3}},
  label={lst:mse-pro}, language=Prolog
]
%! mean_sq_error(+Klassen:list, +Pred:list, -Gen:double) is det
%
%  berechnet den Mean-Squared-Error zwischen Klassen und Pred.
mean_sq_error(Klassen, Pred, Gen) :-
    mean_sq_error(Klassen, Pred, 0, 0, Gen).
mean_sq_error([], [], Len, Acc, Gen) :- !, Gen is Acc / Len.
mean_sq_error([HK|TK], [HP|TP], L, Acc, Gen) :-
    Diff is HK - HP,
    Sq is Diff**2,
    NewAcc is Acc + Sq,
    NewL is L + 1,
    mean_sq_error(TK, TP, NewL, NewAcc, Gen).
\end{lstlisting}
\begin{lstlisting}[
  float, caption={R implementation of \texttt{train.test.regression}},
  label={lst:mse-r}, language=R
]
train.test.regression <- function(trainData, testData) {
  train <- read.csv(trainData)
  tic(trainData)
  fit <- rpart(Y ~ ., train)
  toc(log = TRUE)
  test <- read.csv(testData)
  prediction <- predict(fit, newdata = test, type = "vector")
  mse <- sum((prediction - test$Y)**2) / length(test$Y)
  print(c("Mean-Squared-Error:", mse))
  return(mse)
}
\end{lstlisting}
\begin{lstlisting}[
  float, caption={Python implementation of \texttt{mean\_sq\_err}},
  label={lst:mse-py}, language=Python
]
def mean_sq_err(preds, tests):
    l = len(preds)
    acc = 0
    for i in range(l):
        acc += (preds[i] - tests[i])**2
    return acc / l
\end{lstlisting}

Um mögliche Ausreißer in der Performance auszugleichen werden zu jedem Datensatz fünf
verschiedene Trainings-Test-Aufteilungen durchgeführt und ausgewertet, sodass für diese
Ergebnisse das arithmetische Mittel berechnet werden kann.
Zusätzlich wird für die Zeit der Standard Error of Mean~\cref{eq:sem} berechnet.
Dies ist exemplarisch für den Iris-Datensatz in \cref{lst:iris-tree,lst:err-pro}, für Prolog,
in \cref{lst:iris-r}, für R und in \cref{lst:iris-py,lst:err-py} für Python, implementiert.
\begin{lstlisting}[
  float, caption={Prolog implementation of \texttt{run\_iris\_tree/0}},
  label={lst:iris-tree}, language=Prolog
]
run_iris_tree :-
    train_test_data_tree('Datensets/iris_train1.csv',
                            'Datensets/iris_test1.csv', Time1, Gen1),
    format("-------------------------------------------~n",[]),
    train_test_data_tree('Datensets/iris_train2.csv',
                            'Datensets/iris_test2.csv', Time2, Gen2),
    format("-------------------------------------------~n",[]),
    train_test_data_tree('Datensets/iris_train3.csv',
                            'Datensets/iris_test3.csv', Time3, Gen3),
    format("-------------------------------------------~n",[]),
    train_test_data_tree('Datensets/iris_train4.csv',
                            'Datensets/iris_test4.csv', Time4, Gen4),
    format("-------------------------------------------~n",[]),
    train_test_data_tree('Datensets/iris_train5.csv',
                            'Datensets/iris_test5.csv', Time5, Gen5),
    mean([Time1, Time2, Time3, Time4, Time5],AvgT),
    mean([Gen1, Gen2, Gen3, Gen4, Gen5],AvgG),
    calc_metrics([Time1, Time2, Time3, Time4, Time5], AvgT, _, _, Err),
    format("iris:~n Average train time: ~w sec.~n
                    Standard Error of Mean(Time): ~w ~n
                    Average accuracy: ~w ~n", [AvgT,Err,AvgG]).
\end{lstlisting}
\begin{lstlisting}[
  float, caption={Prolog implementation of \texttt{calc\_metrics/5}},
  label={lst:err-pro}, language=Prolog
]
%! calc_metrics(+Targets:list, +Mean:double, -Varianz:double,
%                                          -Abw:double, -Err:double) is det
%
%  Berechnet die Sample Varianz, Sample Standardabweichung und
%  Standard error of mean
calc_metrics([], _, 0, 0, 0) :- !.
calc_metrics(Targets, Mean, Varianz, Abw, Err) :-
    calc_metrics(Targets, Mean, -1, 0, Varianz, Abw, Err).
calc_metrics([], _, Len, SumSq, Varianz, Abw, Err) :- 
    !,
    Varianz is SumSq / Len,
    Abw is sqrt(Varianz),
    RealLen is Len + 1,
    RootLen is sqrt(RealLen),
    Err is Abw / RootLen.
calc_metrics([H|T], Mean, Len, Acc, Varianz, Abw, Err) :-
    Diff is H - Mean,
    Sq is Diff**2,
    NewAcc is Acc + Sq,
    NewLen is Len + 1,
    calc_metrics(T, Mean, NewLen, NewAcc, Varianz, Abw, Err).
\end{lstlisting}


\begin{lstlisting}[
  float, caption={R implementation of \texttt{run.iris}},
  label={lst:iris-r}, language=R
]
run.iris <- function() {
  print("Training iris: Set1")
  p1 <- train.test.classification('iris_train1.csv', 'iris_test1.csv')
  print('----------------------------------------------------------')
  print("Training iris: Set2")
  p2 <- train.test.classification('iris_train2.csv', 'iris_test2.csv')
  print('----------------------------------------------------------')
  print("Training iris: Set3")
  p3 <- train.test.classification('iris_train3.csv', 'iris_test3.csv')
  print('----------------------------------------------------------')
  print("Training iris: Set4")
  p4 <- train.test.classification('iris_train4.csv', 'iris_test4.csv')
  print('----------------------------------------------------------')
  print("Training iris: Set5")
  p5 <- train.test.classification('iris_train5.csv', 'iris_test5.csv')
  print('----------------------------------------------------------')
  log.lst <- tic.log(format = FALSE)
  tic.clearlog()
  timings <- unlist(lapply(log.lst, function(x) x$toc - x$tic))
  mp <- mean(c(p1,p2,p3,p4,p5))
  variance <- var(timings)
  abw <- sqrt(variance)
  err <- abw / sqrt(length(timings))
  print(c('Average Train Time:', mean(timings)))
  print(c('Standard Error of Mean(Time:)', err))
  print(c('Average Precision:', mp))
  print('--------------------------------------------')
  print('--------------------------------------------')
}
\end{lstlisting}
\begin{lstlisting}[
  float, caption={Python implementation of \texttt{run\_iris}},
  label={lst:iris-py}, language=Python
]
def run_iris():
    print("Training iris: Set1")
    t1, p1 = train_test_classification(iris_feature_train1, iris_target_train1,
                                         iris_feature_test1, iris_target_test1)
    print('----------------------------------------------------------')
    print("Training iris: Set2")
    t2, p2 = train_test_classification(iris_feature_train2, iris_target_train2,
                                         iris_feature_test2, iris_target_test2)
    print('----------------------------------------------------------')
    print("Training iris: Set3")
    t3, p3 = train_test_classification(iris_feature_train3, iris_target_train3,
                                         iris_feature_test3, iris_target_test3)
    print('----------------------------------------------------------')
    print("Training iris: Set4")
    t4, p4 = train_test_classification(iris_feature_train4, iris_target_train4,
                                         iris_feature_test4, iris_target_test4)
    print('----------------------------------------------------------')
    print("Training iris: Set5")
    t5, p5 = train_test_classification(iris_feature_train5, iris_target_train5,
                                         iris_feature_test5, iris_target_test5)
    print('----------------------------------------------------------')
    mt = mean([t1,t2,t3,t4,t5])
    mp = mean([p1,p2,p3,p4,p5])
    var, abw, err = metrics([t1,t2,t3,t4,t5])
    print('Average Train Time:            {t} sec.'.format(t = mt))
    print('Standard Error of Mean(Time):  {err}'.format(err = err))
    print('Average Precision:             {p}'.format(p = mp))
    print()
    print()
\end{lstlisting}
\begin{lstlisting}[
  float, caption={Python implementation of \texttt{metrics}},
  label={lst:err-py}, language=Python
]
def metrics(targets):
    l = len(targets)
    m = mean(targets)
    sumsq = 0
    for i in range(l):
        sumsq += (targets[i] - m)**2
    variance = sumsq / (l - 1)
    abw = variance**(1/2)
    err = abw / (l**(1/2))
    return variance, abw, err
\end{lstlisting}


\subsection{Performance} \label{sec:Performance}
Alle der hier benutzten Datensätze wurden aus der Python Bibliothek scikit-learn
übernommen. Die Erklärungen basieren auf dem jeweiligen \enquote{User Guide} Eintrag auf der Website von scikit-learn.

\subsubsection{Iris}
Bei diesem Datensatz handelt es sich um eine Abwandlung des weit verbreiteten Iris Datensatzes~\cite{fisher1936use}
von Fisher. Im Gegensatz zum Original sind in dem hier benutzten Datensatz keine fehlerhaften
Datenpunkte.
Der Datensatz besteht enthält 3 verschiedene Klassen (0 entspricht \enquote{Iris-Setosa}, 1 entspricht \enquote{Iris-Versicolor},
2 entspricht \enquote{Iris-Verginica}), wovon jede 50 Instanzen besitzt und sich auf ein Typ von Schwertlilien
beziehen. Die Insatzen werden sowohl durch die Kelchblatt Länge und Breite, als auch die
Blütenblatt Länge und Breite in Centimeter beschrieben. 

In \cref{table:tree-iris} erkennt man, dass meine Implementierung, für diesen kleinen Datensatz, die selbe Präzision hat und
noch relativ nah an den Referenz Implementierungen liegt. Es fällt allerdings schon auf,
dass die Ausführung in Sicstus Prolog, im Vergleich zu SWI, deutlich länger braucht.

\begin{table}[ht]
    \begin{center}
      \caption{Entscheidungsbaum Ergebnisse Iris. Die Spalte für die Zeit beinhaltet die durchschnittliche Laufzeit
      und den Standard Error of Mean in Millisekunden. Die Spalte Präzision beinhaltet die durchschnittliche Präzision der Vorhersagen in Prozent}
      \label{table:tree-iris}
      \begin{tabular}{lrr}
        \toprule
        Implementierung        & Zeit                                & Präzision \\
        \midrule
        SWI                 & 26.0  \textpm 6.6 ms                    &  92.88    \\
        Sicstus             & 74.2   \textpm 1.8 ms                    &  92.88    \\
        R                   & < 10   \textpm < 10 ms                   &  92.00    \\
        Python              & 3.1   \textpm  3.1 ms                   &  92.88    \\
        \bottomrule
      \end{tabular}
    \end{center}
\end{table}

In \cref{table:forest-iris} fällt dieser kleiner Abstand stärker ins Gewicht, da jeweils 100 Bäume erstellt werden.
Inzwischen braucht meine Implementierung fast das hundertfache an Zeit im Vergleich zu R, ohne dabei eine höhere Präzision
zu erreichen. Die Ausführung in Sicstus dauert sogar drei mal so Lange wie die in SWI.

\begin{table}[ht]
    \begin{center}
      \caption{Random Forest Ergebnisse Iris. Die Spalte für die Zeit beinhaltet die durchschnittliche Laufzeit
      und den Standard Error of Mean in Millisekunden. Die Spalte Präzision beinhaltet die durchschnittliche Präzision der Vorhersagen in Prozent}
      \label{table:forest-iris}
      \begin{tabular}{lrr}
        \toprule
        Implementierung        & Zeit                                & Präzision \\
        \midrule
        SWI                 & 930.8  \textpm   33.7 ms                    &  92.88    \\
        Sicstus             & 3579.8  \textpm    66.3 ms                    &  93.33    \\
        R                   & 12.0   \textpm     3.7  ms                  &  93.77    \\
        Python              & 115.6   \textpm    3.8  ms                  &  93.77    \\
        \bottomrule
      \end{tabular}
    \end{center}
\end{table}


\subsubsection{Digits}
Dieser Datensatz ist eine Kopie des Testsets von dem UCI ML handgeschriebenen Ziffern Datensatz~\cite{Dua:2019}.
Er enthält die Bilder von handgeschriebenen Ziffern. Somit sind die 10 Zielvariablen die Ziffern
von 0 bis 9. Es sind insgesammt 1797 Bilder vorhanden. Die Bilder wurden so aufbereitet,
dass sie durch eine $8\times8$ Matrix beschrieben werden können, in der jeder Eintrag eine Zahl in dem Intervall von
0 bis 16 ist. Dadurch haben die Instanzen des Datensatzes 64 Attribute.  

\cref{table:tree-digits} zeigt sehr deutlich, dass meine Implementierung für große Datensätze nicht geeignet ist.
Während Python und R im Schnitt unter einer Sekunde bleiben, braucht SWI knapp eine Minute und Sicstus davon mehr als das doppelte.
Überraschenderweise ist die Präzision der R Umsetzung weit abgeschlagen im Vergleich zu den anderen Programmiersprachen.

\begin{table}[ht]
    \begin{center}
      \caption{Entscheidungsbaum Ergebnisse Digits. Die Spalte für die Zeit beinhaltet die durchschnittliche Laufzeit
      und den Standard Error of Mean in Millisekunden. Die Spalte Präzision beinhaltet die durchschnittliche Präzision der Vorhersagen in Prozent}
      \label{table:tree-digits}
      \begin{tabular}{lrr}
        \toprule
        Implementierung        & Zeit                                & Präzision \\
        \midrule
        SWI                 & 58003.0   \textpm   2819.1    ms                &  83.96    \\
        Sicstus             & 159970.0    \textpm    3516.1  ms                  &  83.96    \\
        R                   & 110.0   \textpm     < 0.1         ms           &  75.22    \\
        Python              & 15.6     \textpm    < 0.1      ms              &  83.59    \\
        \bottomrule
      \end{tabular}
    \end{center}
\end{table}

Durch \cref{table:forest-digits} bestätigt sich der Trend. R und Python bleiben unter einer Sekunde, während meine Implementierung über
600 benötigt. Außerdem liegt auch die Präzision leicht unter den anderen Implementierungen, das könnte aber mit der zufälligen
Auswahl der Attribute zusammenhängen.
Da für Sicstus-Prolog nach über 90 Minuten kein Ergebnis vorlag wurde das Experiment abgebrochen.  

\begin{table}[ht]
    \begin{center}
      \caption{Random Forest Ergebnisse Digits. Die Spalte für die Zeit beinhaltet die durchschnittliche Laufzeit
      und den Standard Error of Mean in Millisekunden. Die Spalte Präzision beinhaltet die durchschnittliche Präzision der Vorhersagen in Prozent}
      \label{table:forest-digits}
      \begin{tabular}{lrr}
        \toprule
        Implementierung        & Zeit                                & Präzision \\
        \midrule
        SWI                 & 615226.6   \textpm    14648.1 ms                    &  95.52    \\
        Sicstus             & -                              &  -    \\
        R                   & 480.0      \textpm     12.6   ms                 &  97.37    \\
        Python              & 268.7      \textpm     3.1    ms                &  97.22    \\
        \bottomrule
      \end{tabular}
    \end{center}
\end{table}


\subsubsection{Diabetes}
Der Diabetes Datensatz~\cite{efron2004least} besteht aus den 10 Attributen: \enquote{Alter in Jahren}, \enquote{Geschlecht},
\enquote{Body Mass Index}, \enquote{durchschnittlicher Blutdruck} und sechs verschiedenen Blut Serum Messungen.
Dies sind die gemessenen Attribute von den 442 Diabetes Patienten, die diesen Datensatz representieren.
Dazu kommt die Zielvariable die den Fortschritt der Krankheit nach einem Jahr darstellt.
Alle Attribute sind numerisch. 

In diesen Datensatz fällt auf, dass meine Implementierung, zumindest von dem Mean Square Error her, die Python Implementierung
schlägt. Dadurch, dass der Datensatz wieder kleiner ist, ist die Laufzeit bedeutend kleiner, allerdings immernoch weit entfernt
von R und Python, wie man in \cref{table:tree-diabetes} sehen kann.

\begin{table}[ht]
    \begin{center}
      \caption{Entscheidungsbaum Ergebnisse Diabetes. Die Spalte für die Zeit beinhaltet die durchschnittliche Laufzeit
      und den Standard Error of Mean in Millisekunden. Die Spalte MSE beinhaltet den durchschnittlichen Mean Squared Error der Vorhersagen}
      \label{table:tree-diabetes}
      \begin{tabular}{lrr}
        \toprule
        Implementierung        & Zeit                                & MSE \\
        \midrule
        SWI                 & 1917.2      \textpm    61.4    ms                &  6272.5489    \\
        Sicstus             & 6860.6     \textpm     222.8   ms                 &  6272.5489    \\
        R                   & 4.0    \textpm     4.0         ms           &  4275.9852    \\
        Python              & 3.1      \textpm     3.1       ms             &  6541.9654    \\
        \bottomrule
      \end{tabular}
    \end{center}
\end{table}

\cref{table:forest-diabetes} bestätigt, dass auch im Fall der Regression meine Implementierung
nicht gut skaliert.

\begin{table}[ht]
    \begin{center}
      \caption{Random Forest Ergebnisse Diabetes. Die Spalte für die Zeit beinhaltet die durchschnittliche Laufzeit
      und den Standard Error of Mean in Millisekunden. Die Spalte MSE beinhaltet den durchschnittlichen Mean Squared Error der Vorhersagen}
      \label{table:forest-diabetes}
      \begin{tabular}{lrr}
        \toprule
        Implementierung        & Zeit                                & MSE \\
        \midrule
        SWI                 & 46008.2       \textpm   1958.4 ms                   &  3800.8435    \\
        Sicstus             & 138488.6      \textpm     1704.5 ms                   &  3737.4738    \\
        R                   & 52.0        \textpm     7.3     ms               &  3484.6973    \\
        Python              & 190.6        \textpm     3.1    ms                &  3531.5891    \\
        \bottomrule
      \end{tabular}
    \end{center}
\end{table}


\subsubsection{California Housing}
Der California Housing Datensatz~\cite{pace1997sparse} stammt aus dem StatLib Repository.
Die Zielvariable ist der Median Haus Wert für Districte in Californien, ausgedrückt in
100.000 Dollar. Der Datensatz ist abgeleitet aus dem 1990 U.S Zensus.
Jede Instanz aus dem Datensatz wird durch die Attribute \enquote{Median Einkommen},
\enquote{Median Haus Alter}, \enquote{durchschnittlische Anzahl Räume}, \enquote{durchschnittliche Anzahl Schlafzimmer},
\enquote{Bevölkerung}, \enquote{durchschnittlische Anzahl Hausbewohner}, \enquote{Geographische Breite} und
\enquote{Geographische Länge}.

Im Fall von SWI, als auch Sicstus-Prolog konnte nach 20 Minuten kein Entscheidungsbaum oder Random Forest, auf Grundlage,
der ersten Train-Test-Aufteilung erstellt werden, weswegen, dass Experiment abgebrochen wurde.

\begin{table}[ht]
    \begin{center}
      \caption{Entscheidungsbaum Ergebnisse California. Die Spalte für die Zeit beinhaltet die durchschnittliche Laufzeit
      und den Standard Error of Mean in Millisekunden. Die Spalte MSE beinhaltet den durchschnittlichen Mean Squared Error der Vorhersagen}
      \label{table:tree-california}
      \begin{tabular}{lrr}
        \toprule
        Implementierung        & Zeit                                & MSE \\
        \midrule
        SWI                 & -                        &  -    \\
        Sicstus             & -                            &  -    \\
        R                   & 200.0           \textpm     3.2 ms                   &  0.6052    \\
        Python              & 134.4        \textpm     3.8    ms                &  0.5261    \\
        \bottomrule
      \end{tabular}
    \end{center}
\end{table}
\begin{table}[ht]
    \begin{center}
      \caption{Random Forest Ergebnisse Calinfornia. Die Spalte für die Zeit beinhaltet die durchschnittliche Laufzeit
      und den Standard Error of Mean in Millisekunden. Die Spalte MSE beinhaltet den durchschnittlichen Mean Squared Error der Vorhersagen}
      \label{table:forest-california}
      \begin{tabular}{lrr}
        \toprule
        Implementierung        & Zeit                                & MSE \\
        \midrule
        SWI                 & -                       &  -    \\
        Sicstus             & -                        &  -    \\
        R                   & 16162.0  \textpm     91.7 ms                   &  0.2365    \\
        Python              & 8209.4    \textpm    208.0 ms                   &  0.2481    \\
        \bottomrule
      \end{tabular}
    \end{center}
\end{table}
