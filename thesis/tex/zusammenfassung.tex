\section{Zusammenfassung}

In dieser Arbeit werden zunächst die theoretischen Grundlagen eingeführt, um ein besseres Verständnis für den
Hauptteil zu bekommen. Dafür wird erklärt was ein Entscheidungsbaum ist und, wie der CART-Algorithmus einen erstellen kann,
was ein Random Forest ist und wie es konstruiert und ausgewertet wird und Grundlegende Informationen zu der
Programmiersprache Prolog, welche für den Hauptteil meiner Arbeit benutzt wird.
Im nächsten Kapitel wird dann meine Implementierung für den CART-Algorithmus und die Entscheidungen, die 
ich während der Programmierung getroffen habe, vorgestellt.
Zum Schluss werden zunächst die Grundlagen für den Vergleich der Implementierungen eingeführt und erklärt, um
letztendlich die Ergebnisse von eben diesem vorzustellen.

Das Experiment hat gezeigt, dass meine Implementierung funktioniert.
Die Präzision oder der Mean Squared Error ist dabei stets mit einer Referenz Implementierung
vergleichbar. Allerdings fällt auf, dass je größer der Datensatz ist, meine Implementierung
deutlich mehr Zeit für die Durchführung des Algorithmus braucht.
Dazu kommt, dass die Ausführungszeit mit Sicstus-Prolog bemerkbar länger ist, als die von SWI-Prolog.
Da die Datenmengen, die heutzutage zur Verfügung stehen, immer größer werden, ist es fragwürdig,
ob eine Implementierung die so schlecht mit der Größe des Datensatzes skaliert Anwendung finden kann.

Der große Geschwindigkeitsunterschied lässt sich vielseitig begründen. Zunächst einmal ist die hier vorgestellte Implementation,
im Rahmen einer Bachelorarbeit, also ungefähr in drei Monaten, entstanden. Somit kann der Code nicht zu einem Ausmaß optimiert sein,
wie er es bei den Referenzimplementierungen ist, welche von großen Communitys über einen längeren Zeitraum gepflegt wurden.
Außerdem ist der Alogrithmus an sich nicht optimiert. Ich habe mich in meiner Implementierung an dem grundlegenden CART-Algorithmus
orientiert, jedoch wurden, wie bereits in meiner Arbeit erwähnt, schnellere Verfahren zur Aufteilung des Datensatzes vorgestellt~\cite{MOLA1997}
oder aber auch ein generell optimierter CART-Algorithmus~\cite{CRAWFORD1989197}.
Die verwendeten Programmiersprachen spielen natürlich auch eine Rolle. In dieser wurde motiviert, dass meine Implementierung in puren 
Prolog geschrieben ist. Allerdings benutzt scikit-learn, beziehungsweise numpy, C/C++ im Hintergrund, um die Laufzeit
des Programms deutlich zu verbessern. Im Falle der R Implementierung hat die Programmiersprache den Vorteil, dass sie darauf
ausgelegt ist große Datenmengen effizient zu verarbeiten.

An meiner Implementierung lässt sich also noch einiges machen. Man könnte damit anfangen ein schnelleres Splitting Verfahren zu implementieren
oder auch erstmal kleinschrittig den vorhandenen Code optimieren. Außerdem bietet Prolog eine Schnittstelle zu C und C++. Somit
könnte man die rechenintensiven Teile nach C/C++ auslagern um Zeit zu sparen.