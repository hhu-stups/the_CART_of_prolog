\section{Implementierung}
In diesem Kapitel wird zunächst der selbst angefertigte Quellcode
für den CART Algorithmus ausführlich erklärt. Darufhin folgt der Code für den
Random Forest Algorithmus. Zum Schluss wird noch auf den Code eingegangen, der benötigt wird um
Vorhersagen mit den erzeugten Strukturen zu Treffen.

\subsection{Entscheidungsbäume}
Vorab muss erwähnt werden, dass die Art wie ich meinen Baum und den Datensatz repräsentiere,
auf der Code-Skizze zu Entscheidungsbäumen
aus dem Buch \enquote{Prolog programming for artificial intelligence}~\cite{bratko2001prolog}, basiert.
Wie das aussieht ist in \cref{table:Struktur} dargestellt.
\begin{table}[ht]
  \begin{center}
    \caption{Darstellung von Baum und Datensatz}
    \label{table:Struktur}
    \begin{tabular}{cc}
      \toprule
      Strukur   & Darstellung \\
      \midrule
      Baum      &  \(tree(Attribut = Wert, LinkerTeilbaum, RechterTeilbaum)\)    \\
      Datensatz   &  \([example(Klasse, [Att1 = Wert1,...]), example(...), ...]\)  \\
      \bottomrule
    \end{tabular}
  \end{center}
\end{table}

Analog zu der Erklärung zum Algorithmus, wird die Implementierung von oben nach unten erklärt.
Im ersten Prädikat, \cref{lst:induce-tree}, befindet sich der rekursive Aufruf der den Baum konstruiert.
Der erste Fall gibt einen leeren Baum, beziehungsweise Liste, zurück, falls keine Daten übergeben wurden.
Der zweite Fall überprüft, ob alle momentan genutzten Datenpunkte dieselbe Klasse haben, wie der erste Datenpunkt
in der Liste. Dafür wird das Hilfsprädikat \texttt{check-classes/2} verwendet. Falls dies der Fall sein sollte wird ein Blatt erzeugt.
Der dritte Fall erzeugt zwei Kinder im Entscheidungsbaum. Dafür wird zuerst, mithilfe von \cref{lst:find-best-split},
aus den vorhandenen Datenpunkten, ein Wert zu einem Attribut gefunden, der am besten zwischen den Klassen
differenziert. Daraufhin wird mit Hilfe dieses Attribut-Wert-Paares und \texttt{split/4} der Datensatz entsprechend aufgeteilt.
Der Wert von dem gefundenen Attribut-Wert-Paar ist im numerischen Fall eine Zahl und im kategorischen Fall eine Teilmenge der Werte,
die das entsprechende Attribut annehmen kann.
Dabei landen die Datenpunkte, bei numerischen Werten, die kleiner sind als der gegebene Wert, im linken ansonsten im rechten Teilbaum.
Bei kategorischen Werten landen die Datenpunkte im linken Teilbaum, wenn deren Werte, teil der Teilmenge sind, ansonten im rechten.
Die daraus entstehenden Teilmengen werden dann genutzt um Teilbäume zu konstruieren.
Im Gegensatz zu den üblichen Implementierungen habe ich mich dazu entschieden den Baum bis zur höchsten Reinheit aufzubauen.
Zum einen besteht laut Demidova und Usachev~\cite{Demidova_2020} die Gefahr, dass bei anderen Abbruchbedingungen
die Klassifikations-Qualität verloren geht und zum anderen muss kein extra Quellcode für die Verwendung von Random Forest
angefertigt werden.
Jedoch gibt es eine weitere Abbruchbedingung die eingebaut werden musste. Falls der aktuelle Datensatz Datenpunkte
enthält, die von verschiedenen Klassen sind, sich aber nicht unterscheiden lassen, wird ein Blatt erzeugt.
Bei Klassifikationsbäumen hat das Blatt die Klasse, die am meisten im Datensatz vertreten ist, bei Regressionsbäumen
ist das Blatt das arithmetische Mittel der Zielvariablen im Datensatz.
\begin{lstlisting}[
    float, caption={Prolog implementation of \texttt{induce\_tree/3}},
    label={lst:induce-tree}, language=Prolog
  ]
%! induce_tree(+Daten:list, +Attribute:list, -Tree:term) is det
%
%  induziert den Entscheidungsbaum durch 3 verschiedene Faelle:
%  1) Tree = null, wenn keine Daten existieren
%  2) Tree = leaf(Class), wenn alle Beispiele zur selben Klasse gehoeren
%            oder nicht mehr unterschieden werden koennen
%  3) Tree = tree(Attribut = Wert, SubTree1, SubTree2), wenn es mehr als
%            eine vorhandene Klasse gibt.
induce_tree([], _, []) :- !.
induce_tree([example(Klasse, _)|T], _, leaf(Klasse)) :-
    check_classes(T, Klasse), !.
induce_tree(Daten, Attribute, tree(BestAttVal, SubTree1, SubTree2)) :-
    find_best_split(Daten, Attribute, BestAttVal),
    split(Daten, BestAttVal, Split1, Split2),
    induce_tree(Split1, Attribute, SubTree1),
    induce_tree(Split2, Attribute, SubTree2).
induce_tree(Daten, Attribute, leaf(Klasse)) :-
    find_best_split(Daten, Attribute, err = Klasse), !.
\end{lstlisting}

Als nächstes geht es darum das beste Attribut-Wert-Paar zum aufteilen des Datensatzes zu finden.
Dieser Schritt lässt sich in zwei Teilschritte unterteilen. Als erstes muss eine Liste erstellt werden,
die die Attribut-Wert Paare beinhaltet an denen der Datensatz potentiell aufgeteilt werden kann.
Als zweites muss das Paar ausgewählt werden, welches die zwei \enquote{reinsten} Datensätze hervorbringt.
Die Koordinierung dieser beiden Aufgaben wird von \cref{lst:find-best-split} übernommen.
In der Implementierung wird an dieser Stelle die eben erwähnte Abbruchbedingung abgefangen.
Sollte \texttt{split\_candidates/2} eine leere Kandidatenliste ausgeben, ist klar, dass der aktuelle
Datensatz nicht weiter aufgeteilt werden kann. In dem Fall wird wie beschrieben vorgegangen.
\begin{lstlisting}[
    float, caption={Prolog implementation of \texttt{find\_best\_split/3}},
    label={lst:find-best-split}, language=Prolog
  ]
%! find_best_split(+Daten:list, +Attribute:list, -BestAttVal:Att = Val)
%  is det
%
% Findet den besten Split zur aktuellen Datenlage
find_best_split(Daten, Attribute, BestAttVal) :-
    split_candidates(Daten, Attribute, Kandidaten),
    ([] \= Kandidaten ->
        best_split(Kandidaten, Daten, BestAttVal)
    ;   get_classes(Daten, Klassen),
        sort(Klassen, Klassenliste),
        get_max(Klassenliste, Klassen, Max),
        best_split(Max, err, BestAttVal)).
\end{lstlisting}

Das erstellen der Liste von potentiellen Aufteilungen wird von \texttt{split\_candidates/3} erledigt.
Dieses Prädikat benutzt einen Akkumulator. Das heißt, dass es ein Prädikat mit seinen
Variablen und einer zusätzlichen leeren Liste aufruft. Hier wird mithilfe von \cref{lst:split-candidates-attribute}
eine Liste von Kandidaten für ein einzelnes Attribut erstellt. Diese Liste wird
an den Akkumulator angehangen. Das wird so lange wiederholt, bis alle Attribute abgearbeitet sind.
Dann wird der Akkumulator mit der Ausgabevariablen unifiziert.

Um eine Liste der Kandidaten zu erstellen, geht \cref{lst:split-candidates-attribute} wie folgt vor:
Zunächst wird durch \texttt{check\_attribute\_number/2} überprüft, ob das Attribut numerisch oder kategorisch ist.
Bei numerischen Werten wird \texttt{get\_values\_n/3} aufgerufen.
Hier wird wieder ein Akkumulator benutzt in dem die Werte zu dem aktuellen Attribut eingefügt werden.
Dabei wird darauf geachtet, dass kein Wert, zu einem Attribut, doppelt vorkommt.
Bei kategorischen Werten wird \texttt{get\_values\_k/3} aufgerufen. Hier wird analog
zum vorherigen Prädikat vorgegangen.
Nachdem eine Liste mit unterschiedlichen Werten vorliegt, berechnet \texttt{create\_candidates/3} daraus eine Liste mit möglichen Aufteilungen.
Im numerischen Fall wird dabei die Mitte von zwei Werten aus der übergebenen Liste berechnet und als
Attribut-Wert Paar in die Ergebnisliste aufgenommen. Im kategorischen Fall werden mit Hilfe von \texttt{my\_powerset/2}
mindestens \(2^{m-1} - 1\) verschiedene Kombinationen von Werten aus der Kandidaten Liste erzeugt, wobei m die Anzahl an Werten ist,
die das Attribut annehmen kann.
Dabei ist jede Kombination in einer Liste verpackt und wird dann als Attribut-Liste-Paar
in die Ergebnisliste aufgenommen.
Dadurch beinhaltet die Liste alle möglichen Kandidaten zur Aufteilungen des aktuellen Datensatzes.
Alternativ hätte man bei numerischen Werten das Minimum und Maximum nehmen können und in diesem Intervall
eine feste Anzahl an Aufteilungen, in gleichmäßigen Abständen durchführen können~\cite{breiman1984classification}.
Mola \cite{MOLA1997} hat auch schon einen schnelleren Aufteilungsalgorithmus vorgestellt, der auf dem Vorhersehbarkeits
Index basiert.

\begin{lstlisting}[
    float, caption={Prolog implementation of \texttt{split\_candidates\_attribute/3}},
    label={lst:split-candidates-attribute}, language=Prolog
  ]
%! split_candidates_attribute(+Daten:list, +Attribut:atom, -Kandidaten:list)
%  is det
%
%  Erstellt ein Liste mit allen moeglichen Splits zu einem Attribut
split_candidates_attribute(Daten, Attribut, Kandidaten) :-
    (check_attribute_number(Daten, Attribut) -> 
    get_values_n(Daten, Attribut, Values)
    ; get_values_k(Daten, Attribut, Values)),
    create_candidates(Values, Attribut, Kandidaten).
\end{lstlisting}

Mit der Liste aller möglichen Aufteilungen wird durch \cref{lst:best-split} die Aufteilung vom Datensatz gefunden,
die das Maß der Unreinheit minimiert. Im ersten Fall wird die Abbruchbedingung abgefangen und die Klasse für das Blatt nach oben gegeben.
Ansonsten werden die Unreinheit der Aufteilung, die durch den ersten
Kandidaten entsteht, und der Kandidat selbst, je als Variable gemerkt. Danach wird in jedem Schritt die Unreinheit
eines neuen Kandidaten berechnet und mit der entsprechenden Variable verglichen. Wenn der Wert kleiner ist,
wird mit diesem in den nächsten Schritten verglichen und der Kandidat wird auch gemerkt. Falls der neue Wert
der Unreinheit größer ist, werden die Variablen nicht überschrieben und es geht mit dem nächsten Kandidaten weiter.

\begin{lstlisting}[
  float, caption={Prolog implementation of \texttt{best\_split/3}},
  label={lst:best-split}, language=Prolog
]
%! best_split(+Kandidaten:list, +Daten:list, -BestAttVal: Att = Val)
%  is det
%! Kandidaten = [Att = Val1, Att = Val2, ...]
%
%  Verwendet alle Kandidaten aus der Liste um den besten Split
%  zu finden und auszugeben
best_split(Klasse, err, err = Klasse).
best_split([H|T], Daten, Res) :-
    impurity(Daten, H, Min),
    best_split(T, Daten, Min, H, Res).
best_split([], _, _, Res, Res) :- !.
best_split([H|T], Daten, Min, Cur, Res) :-
    impurity(Daten, H, CombUnreinheit),
    (CombUnreinheit < Min ->
        best_split(T, Daten, CombUnreinheit, H, Res)
    ;   best_split(T, Daten, Min, Cur, Res)).
\end{lstlisting}

Entscheidend ist auch, welches Maß der Unreinheit verwendet wird.
Dieses wird hinter dem Prädikat \cref{lst:impurity} versteckt.
\begin{lstlisting}[
  float, caption={Prolog implementation of \texttt{impurity/3}},
  label={lst:impurity}, language=Prolog
]
%! impurity(+Daten:list, +AttVal:Attribute = Value, -Unreinheit:double) is  det
%
%  Gibt die kombinierte Unreinheit, von dem Split des Datensatzes, aus
impurity(Daten, AttVal, CombUnreinheit) :-
    split_class(Daten, AttVal, Split1, Split2),
    combgini(Split1, Split2, CombUnreinheit).
\end{lstlisting}
In dieser Implementierung wird die \enquote{Gini-Unreinheit} verwendet.
Doch bevor gerechnet werden kann, bereitet \texttt{split-class/3} die Daten entsprechend auf.
Dieses Prädikat erstellt aus dem übergebenen Datensatz, mithilfe eines Attribut-Wert Paars, zwei kleinere Datensätze.
Die kleineren Datensätze bestehen dabei nur aus den Klassen der Datenpunkte, die in dem Datensatz zugeordnet wurden.
Dafür werden wieder 2 Akkumulatoren verwendet. Im ersten Fall wird zunächst nach dem Attribut vom Attribut-Wert-Paar
im Datensatz gesucht. Sobald dieses gefunden ist, gibt es einen Fall dafür, dass der Wert zu diesem
Attribut kategorisch oder numerisch ist. Im numerischen Fall wird überprüft, ob der Wert des Datenpunktes kleiner ist,
als der des übergebenen Werts. Falls dies der Fall ist, wird die Klasse des Datenpunktes an den linken Akkumulator angehangen,
ansonsten am rechten. Analog dazu wird mit kategorischen Werten umgegangen, nur das hier überprüft wird, ob
der Wert des Datenpunkts in der Liste der übergebenen Werte vorhanden ist.

Nach der Erstellung, der aus Klassen bestehenden Datensätze, berechnet \cref{lst:combgini}
die kombinierte Gini-Unreinheit der beiden Datensätze. Als erstes wird die Gini-Unreinheit der
einzelnen Datensätze mit \texttt{gini/2} berechnet. Die Ergebnisse daraus werden dann mit dem Anteil
an Datenpunkten, den der jeweilige Datensatz aus dem ursprünglichen Datensatz hatte, multipliziert.
Diese Ergebnisse werden dann addiert und bilden die kombinierte Unreinheit.
Dieses Verfahren entspricht \cref{eq:combgini}.

\begin{lstlisting}[
  float, caption={Prolog implementation of \texttt{combgini/3}},
  label={lst:combgini}, language=Prolog
]
%! combgini(+Split1:list, +Split2:list, -CombUnreinheit:double) is det
%
%  Berechnet die gewichtete Summe von zwei Splits.
%  Beachte Split ist eine Liste von Klassen!
combgini(Split1, Split2, CombUnreinheit) :-
    gini(Split1, Unreinheit1),
    gini(Split2, Unreinheit2),
    length(Split1, LSplit1),
    length(Split2, LSplit2),
    LDaten is LSplit1 + LSplit2,
    Gewicht1 is LSplit1 / LDaten,
    Gewicht2 is LSplit2 / LDaten,
    GUnreinheit1 is Gewicht1 * Unreinheit1,
    GUnreinheit2 is Gewicht2 * Unreinheit2,
    CombUnreinheit is GUnreinheit1 + GUnreinheit2.
\end{lstlisting}

Der Algorithmus für die Regression funktioniert zu weiten Teilen wie der Klassifikationalgorithmus,
nur dass hier, als Maß der Unreinheit, die Summe der quadrierten Residuen minimiert wird.
\cref{lst:residual-sum-sq} ruft \texttt{mean/2} auf um das arithmetische Mittel,
der Werte in den Datensätzen zu berechnen. Damit wird dann in \texttt{residual-split/3}
die Summe der quadrierten Differenzen, der einzelnen Werte, zum arithmetischen Mitteln, ihres
jeweiligen Datensatzes, berechnet. Dieses Verfahren entspricht \cref{eq:rss}
Zum Schluss werden die beiden Summen addiert und ergeben das Maß der Unreinheit.

\begin{lstlisting}[
  float, caption={Prolog implementation of \texttt{residual\_sum\_sq/3}},
  label={lst:residual-sum-sq}, language=Prolog
]
%! residual_sum_sq(+Split1:list, +Split2:list, -ResSumSq:double) is det
%
%  Berechnet die Summe der quadrierten Residuen von zwei Splits.
%  Beachte Split ist eine Liste von Target Werten!
residual_sum_sq(Split1, Split2, ResSumSq) :-
    mean(Split1, Mean1),
    mean(Split2, Mean2),
    residual_split(Split1, Mean1, SumSq1),
    residual_split(Split2, Mean2, SumSq2),
    ResSumSq is SumSq1 + SumSq2.
\end{lstlisting}

Um keinen Code programmieren zu müssen, der eine Wahl ermöglicht, ob ein Klassifikationsbaum
oder Regressionsbaum erstellt werden soll, sind beide Verfahren in eigenen Dateien gespeichert,
sodass man nur das entsprechende Modul benutzten muss, den dazugehörigen Baum zu erzeugen.

\subsection{Random Forest}

Mein Random-Forest-Algorithmus besteht aus drei Schritten. Als erstes werden zufällig Datenpunkte aus dem Datensatz,
mit zurücklegen gezogen. Dadurch werden Datenpunkte entsprechend der Anzahl im Datensatz gezogen.
Als zweites wird eine Liste von zufälligen Teilmengen der Attribute erzeugt. Als drittes wird aus jeweils
einem gezogenen Datensatz und einer Teilmenge von Attributen ein Entscheidungsbaum erzeugt.
Sowohl die Anzahl der gezogenen Datensets und der Teilmengen von Attributen werden im voraus bestimmt
und resultieren in einer Liste mit dieser Anzahl an Entscheidungsbäumen.
Dieser Vorgang wird durch \cref{lst:induce-forest} koordiniert.

\begin{lstlisting}[
  float, caption={Prolog implementation of \texttt{induce\_forest/3}},
  label={lst:induce-forest}, language=Prolog
]
%! induce_forest(+Daten:list, Attribute:list, N:int,-RandomForest:list) is det
%
%  erstellt aus den Daten einen Entscheidungswald;
%  Die Anzahl der Baueme wird angegeben;
%  die Anzahl pro Schritt benutzten Attribute orientiert sich 
%  an einer in der praxis bewaehrten Faustregel
induce_forest(Daten, Attribute, N, RandomForest) :-
    bootstrappen(Daten, N, Samples),
    choose_attributes(Attribute, N, AttributSets),
    induce_trees(Samples, AttributSets, RandomForest).
\end{lstlisting}

Als nächstes werden die benötigten Prädikate im Detail erklärt.

Das zufällige erstellen von Datensätzen wird durch \cref{lst:bootstrappen} implementiert.
Zunächst wird die Länge des Datensatzes bestimmt. Danach werden durch \texttt{get\_sample/3},
entsprechend dieser Anzahl zufällig Datenpunkte, mit zurücklegen, gezogen. Das Ergebnis wird als
Liste in eine Liste eingefügt. Dieser Vorgang wird so oft wiederholt, wie Entscheidungsbäume am Ende
erzeugt werden sollen.

\begin{lstlisting}[
  float, caption={Prolog implementation of \texttt{bootstrappen/3}},
  label={lst:bootstrappen}, language=Prolog
]
%! bootstrappen(+Daten:list, +Trees:int, -Samples:list) is det
%
%  erstellt eine Liste von Listen, wobei jede der N Listen
%  ein bootstrap sample von dem Datensatz ist
bootstrappen(Daten, N, Samples) :-
    length(Daten, Len),
    bootstrappen(Daten, Len, N, Samples).
bootstrappen(_, _, 0, []) :- !.
bootstrappen(Daten, Len, N, [Sample|Samples]) :-
    get_sample(Daten, Len, Sample),
    NewN is N - 1,
    bootstrappen(Daten, Len, NewN, Samples).
\end{lstlisting}

Die Erstellung der Teilmengen der Attribute wird durch \cref{lst:choose-attributes} initiiert.
Hier wird die Größe der Teilmengen bestimmt. Die Anzahl der benutzten Attribute
wird, wie in \cref{eq:att-klass,eq:att-reg} vorgestellt berechnet.

\begin{lstlisting}[
  float, caption={Prolog implementation of \texttt{choose\_attributes/3}},
  label={lst:choose-attributes}, language=Prolog
]
%! choose_attributes(+Attribute:list, +N:int, -AttributSets) is det
%
%  erstellt N Listen, welche aus zufaellig ausgewahelten Attributen bestehen.
%  Die Anzahl der Ausgewaehlten Attributen ist,
%  bei Klassifikation, die abgerundete Wurzel, der Anzahl, der Attribute
choose_attributes(Attribute, N, AttributSets) :-
    length(Attribute, Int),
    TRes is sqrt(Int),
    Res is floor(TRes),
    choose_all_members(Attribute, Res, N, AttributSets).
\end{lstlisting}

Das eigentliche Erstellen der Teilmengen ist in \cref{lst:choose-all-members} implementiert.
Hier wird in jeden Schritt eine Permutation der Attribut-Liste erstellt. Anschließend
werden so viele Attribute wie benötigt, vom Anfang der Liste beginnend herausgenommen und als
Teilmenge in die Ergebnisliste aufgenommen.
Dieser Vorgang wird wiederholt, bis die Anzahl der angefertigten Teilmengen,
der vorgegebenen Anzahl entspricht.

\begin{lstlisting}[
  float, caption={Prolog implementation of \texttt{choose\_members/4}},
  label={lst:choose-all-members}, language=Prolog
]
%! choose_all_members(+Attribute:list, +Len:int, +N:int, -AttributSets:list) is det
%
%  erstell N, Len Lange Listen, welche, zufaellig ausgewaehlte, unterschiedliche
%  Elemente aus Attribute beinhalten
choose_all_members(_, _, 0, []) :- !.
choose_all_members(Attribute, Len, N, [AttributSet|AttributSets]) :-
    random_permutation(Attribute, Permutation),
    take(Permutation, Len, AttributSet),
    NewN is N - 1,
    choose_all_members(Attribute, Len, NewN, AttributSets).
\end{lstlisting}

Zum Schluss wird von \texttt{induce\_trees/3} eine Liste von Entscheidungsbäumen erstellt.
Dafür wird jeweils ein zufällig erstellter Datensatz und eine Teilmenge der Attribute genommen
und \cref{lst:induce-tree} aufgerufen.

\subsection{Vorhersagen}
In diesem Unterkapitel wird der Quellcode erklärt, der benutzt wird um eine
Vorhersage mit einem Entscheidungsbaum oder einem Random Forest zu treffen.
Bei der Vorhersage die von Entscheidungsbäumen getroffen werden, überprüft \cref{lst:check-sample},
ob der Wert mit dem getrennt wird numerisch oder kategorisch ist. Im numerischen Fall,
wird überprüft, ob der Wert vom Datenpunkt, zum entsprechendem Attribut kleiner ist,
als der im Knoten. Falls ja, geht es in dem linken Teilbaum weiter, ansonsten im rechten.
Der kategorische Fall funktioniert analog, nur das überprüft wird ob, der Wert des Datenpunktes,
ein Element in der Liste der Werte ist.
Sobald man in einem Blatt angekommen ist, ist der Inhalt des Blatts die Vorhersage.

\begin{lstlisting}[
  float, caption={Prolog implementation of \texttt{check\_sample/3}},
  label={lst:check-sample}, language=Prolog
]
%! check_sample(+Tree:Entscheidungsbaum, +Example:list, -Res:Class) is det
%
%  Gibt die vom Entscheidungsbaum vorhergesehene Klasse zu einem Datenpunkt
%  ([att1=val1,...]) aus
check_sample(leaf(Class), _, Class).
check_sample(tree(Att = Val1, SubTree1, SubTree2), Example, Result) :-
    get_value(Example, Att, Val2),
    (is_list(Val1) ->
        (member(Val2,Val1) ->
            check_sample(SubTree1,Example,Result)
        ;   check_sample(SubTree2,Example,Result))
    ;   (Val2 < Val1 ->
            check_sample(SubTree1,Example,Result)
        ;   check_sample(SubTree2,Example,Result))).
\end{lstlisting}

Um mehrere Datenpunkte auf einmal einzugeben wurde \texttt{check\_all\_sample/3}
implementiert. Dieses Prädikat ruft für jeden Datenpunkt \cref{lst:check-sample} auf
und verpackt alle Ergebnisse in einer Liste.

Ein Random Forest, trifft seine Vorhersage, durch einen Mehrheitsentscheid oder das arithmetische Mittel.
In der Implementierung wird für einen Datenpunkt eine Vorhersage von jedem Entscheidungsbaum im 
Random Forest getroffen.
Im Fall der Klassifikation wird die am meisten vorhergesagte Klasse ausgegeben, im Fall der Regression,
wird das arithmetische Mittel aller Vorhersagen ausgegeben.