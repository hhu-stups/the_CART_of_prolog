Diese Arbeit beschäftigt sich mit der Frage, wie eine Implementierung des CART-Algorithmus und ein darauf aufgebauter
Random-Forest-Algorithmus, in der logischen Programmiersprache Prolog, aussehen könnte.
Zusätzlich wird auch die Schnelligkeit und Genauigkeit
dieser Implementierung mit den Implementierung aus der Python-Bibliothek scikit-learn und den R-Bibliotheken
rpart und randomForest verglichen.
Prolog ist eine beliebte Programmiersprache, die vorallem im Bereich der künstlichen Intelligenz
gerne verwendet wird. Allerdings existiert nur wenig Unterstützung für die Sprache, um Algorithmen,
aus dem Bereich des maschinellen Lernens, einem Teilgebiet des künstlichen Intelligenz, direkt anzuwenden.
Es wird sich leider herausstellen, dass diese Implementierung noch nicht ausgereift genug ist, um 
im Rahmen der Schnelligkeit mit den Referenzimplementierung mithalten zu können.